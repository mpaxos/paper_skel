% -*- mode: latex -*-
% This is stuff that is required for the document to compile. It is not
% related to formatting. 
\usepackage[pdftex]{graphicx}
\usepackage{booktabs}  % for tables
\usepackage{dcolumn}   % for tables
%\usepackage{hhline}   % for tables
\usepackage{multirow}  % for tables
\usepackage{rotating}
\usepackage{xspace}
\usepackage{hyphenat}  % supplies \hyp{}, which tells tex that it can 
		       % hyphenate at an existing hyphen
\usepackage{color}
\usepackage{float}                           % Must appear before hyperref to
                                             % avoid weird PDF compile issues
\usepackage[dvipsnames]{xcolor}
\usepackage{enumerate}
\usepackage{textpos}
\usepackage{balance} % making the last page balanced
\usepackage{multirow}
\usepackage[symbol*]{footmisc}
\usepackage{mathtools}

%\usepackage[square,comma,numbers,sort&compress]{natbib}

%\usepackage{ulem}      % for strikethrough and underlining
%\mw{above package turns italics to underlines.}

\usepackage{wrapfig}
%\usepackage{textcomp}
\usepackage{lastpage}
\usepackage{tabularx}
\usepackage{pifont}
\usepackage{adjustbox} % textbox as figures
%\usepackage{natbib}
%\usepackage{url}

\usepackage{wrapfig}

\usepackage{ulem}
\normalem


% for getting table 1 in hotnets09 submission to work
\usepackage{colortbl} % colorful columns and rows
\usepackage{array}    % needed for 'b' argument in tabular preamble

\usepackage{dblfloatfix}

%%%
%%%  Captions
%%%
\usepackage[font=bf]{caption}
\usepackage{subcaption}

%%%
%%% SM: select the algorithm package you want to use
%%%
%\usepackage{algorithm}
%\usepackage{algorithmic}
%\floatname{algorithm}{Pseudocode}
%\usepackage[linesnumbered,noresetcount,ruled,vlined]{algorithm2e}
%\usepackage{algpseudocode}

%\algrenewcomment[1]{\hfill// #1}%
% Hack algpseudocode to be more Python-like
%\algnotext{EndFunction}
%\algnotext{EndFor}
%\algnotext{EndIf}
%\algnotext{EndWhile}
%\algnotext{EndLoop}
%\algrenewcommand{\algorithmicdo}{\hspace{-0.2em}:}
%\algrenewcommand{\algorithmicthen}{\hspace{-0.2em}:}
%\algrenewcommand{\algorithmiccomment}[1]{{\color{CadetBlue}\hfill// #1}}

\usepackage{amsmath,amscd}
\usepackage{amssymb}
\usepackage{amsfonts}
\usepackage{amsthm}
\usepackage[dvipsnames]{xcolor}
\usepackage{tikz}
\usetikzlibrary{shapes.misc,positioning,patterns}
%\usepackage{mathrsfs}
%\usepackage{minted}
%\usemintedstyle{xcode}
%\usepackage[noeka]{mathrmletter}

\ifthenelse{\equal{\COLOR}{yes}}{%
  \usepackage[colorlinks]{hyperref}%         % for online version
}{%
  \usepackage[pdfborder={0 0 0}]{hyperref}%  % for paper (B&W) version
}
\usepackage{url}
\usepackage{cleveref}


\renewcommand{\ttdefault}{cmtt}
\newcommand{\mintcpp}[1]{\mintinline{cpp}{#1}}

%\newif\iftechreport
%\newif\ifproof
%\newif\ifextended
%\newif\iflongbatching
%\newif\ifsubmission
%\newif\ifelementary
%
%\ifx\buildextended\undefined
%\else
%    \extendedtrue
%\fi
%\ifx\buildproof\undefined
%\else
%  \prooftrue
%\fi
%\techreporttrue
%\submissiontrue

% Default colors to bright
% MW commenting out. this is like drawing attention to punctuation.
% makes no sense. it's not WEKWD (what eddie kohler would do).
%\definecolor{MyCiteColor}{rgb}{0,0.4,0}
%\definecolor{MyLinkColor}{rgb}{0,0,0.4}

% peanut gallery comments
% NOTE: Comment out the line below if you want a draft with no red comments.
% NOTE: Commenting out this line may replace some of the red comments with 
%       extra spaces or newlines.
%
\newcommand{\zstar}{{$^{\star}$}\xspace}
\newcommand{\zdag}{{$^{\dagger}$}\xspace}
\newcommand{\zddag}{{$^{\ddagger}$}\xspace}
\newcommand{\zsect}{{$^{\S}$}\xspace}
\newcommand{\zparallel}{{$^{\parallel}$}\xspace}
%

\newcommand{\textred}[1]{\textcolor{red}{#1}}
%\newcommand{\textcolor}[2]{\begingroup \color{#1} #2\endgroup}
\ifx\noeditingmarks\undefined
   \newcommand{\pgwrapper}[3]{\begingroup \color{#1} #2: #3 \endgroup}
   \newcommand{\pgwrapperb}[1]{\textbf{#1}}
\else
   \newcommand{\pgwrapperb}[1]{}
   \newcommand{\pgwrapper}[3]{}
\fi
\newcommand{\jl}[1]{\pgwrapper{red}{JL}{#1}}
\newcommand{\mw}[1]{\pgwrapper{green}{MW}{#1}}
\newcommand{\sm}[1]{\pgwrapper{purple}{SM}{#1}}
% end peanut gallery comments

% definitions for MP
\def\hn{\usefont{OT1}{phv}{mc}{n}\selectfont}
\def\hb{\usefont{OT1}{phv}{bc}{n}\selectfont}
\newcommand{\mpfont}{\hn\scriptsize}

\ifx\noeditingmarks\undefined
    \newcommand{\MPworker}[2]{{\color{#1}\vrule\vrule}{\marginpar{\color{#1}\mpfont #2}}}
\else
    \newcommand{\MPworker}[2]{}
\fi

\newcommand{\JL}[1]{\MPworker{red}{MW: #1}}
\newcommand{\SP}[1]{\MPworker{blue}{SGA: #1}}
\newcommand{\ST}[1]{\MPworker{green}{SM: #1}}
\newcommand{\MPcheck}[1][]{\MPworker{blue}{CHECK #1}\xspace}

\ifx\noeditingmarks\undefined
    \newcommand{\changebars}[2]{%
    [{\color{magenta}\em \begingroup {#1} \endgroup}][{\color{magenta}\sout{#2}}]}
\else
    \newcommand{\changebars}[2]{#1}
\fi

\newcommand{\changebarsii}[2]{#1}

 
% ref appendix in different styles, depending on whether it's the
% extended version and depending on what the text is trying to do.
\newcommand{\refappendixplain}[1]{%
\ifextended%
Appendix~#1%
\else%
%Appendix #1~\cite{extended_version}%
Appendix #1~\cite{braun13verifyingextended}%
\fi%
\xspace
}

\newcommand{\refappendixinside}[2][shown in]{%
\ifextended%
\ (#1 Appendix~#2)%
\else%
%~\cite[Appendix #2]{extended_version}%
~\textred{[TR, Apdx. #2]}
\fi%
\xspace
}

\setlength{\marginparwidth}{15mm}
\setlength{\marginparsep}{0.35mm}

\newcommand{\sha}{{\mathrm{SHA{\mathchar"002D}1}}}
\newcommand{\hyph}{{\mathchar"002D}}

\newcommand{\sys}{Mesto\xspace} 
\newcommand{\Sys}{\sys}

\theoremstyle{definition}
\newtheorem{theorem}{Theorem}
\newtheorem{definition}{Definition}
\newtheorem{assumption}{Assumption}
\newtheorem{lemma}{Lemma}
\newtheorem{claim}[lemma]{Claim} 
\newtheorem{corollary}[lemma]{Corollary}
%\newenvironment{proof}[1][Proof]{\begin{trivlist}
%\item[\hskip \labelsep {\bfseries #1}]}{\end{trivlist}}


% customize thanks symbols
\makeatletter
\renewcommand*{\@fnsymbol}[1]{\ensuremath{\ifcase#1\or \star\or \dagger\or \ddagger\or
   \mathsection\or \mathparagraph\or \|\or **\or \dagger\dagger
   \or \ddagger\ddagger \else\@ctrerr\fi}}
\makeatother

\newcommand{\circledone}{\ding{192}\xspace}
\newcommand{\circledtwo}{\ding{193}\xspace}
\newcommand{\circledthree}{\ding{194}\xspace}
\newcommand{\circledfour}{\ding{195}\xspace}
\newcommand{\circledfive}{\ding{196}\xspace}
\newcommand{\filledone}{\ding{202}\xspace}
\newcommand{\filledtwo}{\ding{203}\xspace}
\newcommand{\filledthree}{\ding{204}\xspace}
\newcommand{\filledfour}{\ding{205}\xspace}
\newcommand{\filledfive}{\ding{206}\xspace}

% latex shortcuts. No notation hard-coded here.
\newcommand{\A}{\mathcal{A}}
\newcommand{\C}{\mathcal{C}}
\newcommand{\E}{\mathcal{E}}
\newcommand{\Z}{\mathbb{Z}}
\newcommand{\Q}{\mathbb{Q}}
\newcommand{\F}{\mathbb{F}}
\newcommand{\V}{\mathcal{V}}
\newcommand{\Vpcp}{\mathcal{V}_\text{pcp}}
\newcommand{\R}{\mathcal{R}}
\newcommand{\Pb}{\mathcal{P}}
\newcommand{\Vsc}{\V_\mathrm{SC}}
\newcommand{\Pbsc}{\Pb_\mathrm{SC}}
\newcommand{\ha}{\hat\alpha}
\newcommand{\tf}{\tilde{f}}
\newcommand{\vq}{\vec{q}}
\newcommand{\ext}{{\small\textsf{Ext}}}
\newcommand{\argmax}{\operatornamewithlimits{argmax}}
\newcommand{\myvec}[1]{\mathbf{#1}}
\newcommand{\prover}{prover\xspace}
\newcommand{\verifier}{verifier\xspace}
\newcommand{\Cbound}{\C(X{=}x,Y{=}y)}
\newcommand{\multidecommit}{{\small\textsf{Commit+Multidecommit}}\xspace}
\newcommand{\batchedmultidecommit}{{\small\textsf{BatchedCommit+Multidecommit}}\xspace}
\newcommand{\keygen}{{\small\textsf{KeyGen}}}
\newcommand{\commit}{{\small\textsf{Commit}}}
\newcommand{\gensmall}{{\small\textsf{Gen}}}
\newcommand{\encsmall}{{\small{Enc}}}
\newcommand{\decsmall}{{\small{Dec}}}
\newcommand{\gen}{{Gen}}
\newcommand{\enc}{{Enc}}
\newcommand{\dec}{{Dec}}
%
\newcommand{\dist}[1]{{\small\textsf{Dist}(#1)}}
\newcommand{\rej}[1]{{\small\textsf{Rej}(#1)}}
\newcommand{\smulsmall}{{\small\textsf{HMul}}}
\newcommand{\psmulsmall}{{\small\textsf{PHMul}}}
\newcommand{\haddsmall}{{\small\textsf{HAdd}}}
\newcommand{\phaddsmall}{{\small\textsf{PHAdd}}}
\newcommand{\bintreereduce}{{\small\textsf{BintreeReduce}}}
\newcommand{\mmul}{{\small\textsf{SSLShaderModMul}}}
\newcommand{\mexp}{{\small\textsf{SSLShaderModExp}}}
\newcommand{\ip}[2]{\left\langle#1,#2\right\rangle}
\newcommand{\ipsmall}[2]{\langle#1,\,#2\rangle}
\newcommand{\op}[2]{#1\otimes#2}
\newcommand{\opsub}[3]{#1\otimes_#3 #2}
\newcommand{\vp}[2]{#1\oslash#2}
\newcommand{\ceil}[1]{\lceil#1\rceil}
\newcommand{\floor}[1]{\lfloor#1\rfloor}
\newcommand{\usec}{$\mu\mathrm{s}$}
\newcommand{\accept}{\texttt{accept}\xspace}
\newcommand{\reject}{\texttt{reject}\xspace}
\newcommand{\bigproof}{\boldsymbol{\pi}}

\newcommand{\lcm}{\operatorname{lcm}}

\newcommand{\tV}{\tilde{V}}
\newcommand{\tE}{\tilde{E}}
\newcommand{\add}{\mathrm{add}}
\newcommand{\mult}{\mathrm{mult}}
\newcommand{\tadd}{\tilde{\add}}
\newcommand{\tmult}{\tilde{\mult}}
\newcommand{\qeq}{\stackrel{?}{=}}
\newcommand{\dEq}{d_\mathrm{eq}}
\newcommand{\GEq}{G_\mathrm{eq}}
\newcommand{\Gck}{G_\mathrm{ck}}
\newcommand{\Gcp}{G_{\Psi,\text{prop}}}
\newcommand{\Gcpck}{G_{\Psi,\text{ck}}}
%\newcommand{\dEq}{d_\mathrm{cmp}}
%\newcommand{\GEq}{G_\mathrm{cmp}}

\newcommand{\Xlt}{X^{(\ttlt)}}
\newcommand{\Xne}{X^{(\ttne)}}
\newcommand{\Mlt}{M^{(\ttlt)}}
\newcommand{\Mne}{M^{(\ttne)}}

\newcommand{\defeq}{\stackrel{\text{def}}{=}}

% constants/parameters for Pantry. first decide what kind of thing they
% are.
\newcommand{\constant}{constant\xspace}
\newcommand{\chandley}{c_1}
\newcommand{\ccreatequery}{c_2}
\newcommand{\cperinsttests}{c_3}
\newcommand{\cperinputtests}{c_4}
\newcommand{\creplyquery}{c_5}
\newcommand{\cproofvector}{c_6}
\newcommand{\cquerylength}{c_7}
\newcommand{\creplylength}{c_8}

\newcommand{\Ch}{\C_{\hash}}
\newcommand{\Chinv}{\C_{\hashinv}}
\newcommand{\Cmapper}{\C_{\text{Mapper}}}
\newcommand{\Creducer}{\C_{\text{Reducer}}}
\newcommand{\Cload}{\C_{\text{Load}}}
\newcommand{\Cstore}{\C_{\text{Store}}}

\newcommand{\mapout}{\textit{map\_out}}
\newcommand{\reducein}{\textit{reduce\_in}}
\newcommand{\chunk}{\textit{ch}}
\newcommand{\Tmap}{T_\text{mapper}}
%\newcommand{\cmap}{c_\text{map}}
\newcommand{\Zmapper}{Z_\text{mapper}}
\newcommand{\Tmapper}{T_\text{mapper}}

\newcommand{\digest}{digest\xspace}

\newcommand{\bigconstant}{10^9}
\newcommand{\bigconstanttwo}{96}
\newcommand{\bigexponent}{6}

\newcommand{\orderssavedbythreerefinements}{17\xspace}
\newcommand{\orderssavedbyfourrefinements}{20\xspace}
%\newcommand{\orderssavedbyoliveprime}{17}\MP{check}\xspace}

\newcommand{\empiricalbatchmm}{\textred{50}}
\newcommand{\empiricalbatchdii}{\textred{50}}
\newcommand{\empiricalbatchdiii}{\textred{50}}

\newcommand{\mc}[1]{\mathcal{#1}}

% Here is where our notation begins. Note, please see file NOTATION
\newcommand{\z}{z}
\newcommand{\bcircuit}{\mathcal{C}}
%\newcommand{\poly}{P}
%\newcommand{\prewf}{\Phi}
\newcommand{\asstmm}[3]{z^{#1}_{#2,#3}}
\newcommand{\varmm}[3]{Z^{#1}_{#2,#3}}
\newcommand{\randommm}[3]{v^{#1}_{#2,#3}}
\newcommand{\field}{\F}
\newcommand{\fracfield}[1]{\mathrm{Frac}({#1})}
\newcommand{\zmodp}{\mathbb{Z}/p}
\newcommand{\qmodp}{\mathbb{Q}/p}
\newcommand{\distq}{q}
\newcommand{\negl}{\textrm{neg}}
\newcommand{\rep}{\rho} 
\newcommand{\nbatch}{\beta} 
\newcommand{\nbatches}{\nbatch} 
%\newcommand{\nconst}{\chi}
\newcommand{\nconst}{{|\C|}}
\newcommand{\nqueriespepper}{\approx1000}
\newcommand{\nquerypepper}{\nqueries}
\newcommand{\nqueries}{\mu}
\newcommand{\conq}{t}  % consistency query
%\newcommand{\pcpkappa}{6/7}

% QAP stuff
\newcommand{\poly}[2]{#1(#2)}
\newcommand{\polyvar}{t}
\newcommand{\apolynotat}{A}
\newcommand{\bpolynotat}{B}
\newcommand{\cpolynotat}{C}
\newcommand{\divpolynotat}{D}
\newcommand{\hpolynotat}{H}
\newcommand{\apoly}[2][\polyvar]{\poly{\apolynotat_{#2}}{#1}}
\newcommand{\bpoly}[2][\polyvar]{\poly{\bpolynotat_{#2}}{#1}}
\newcommand{\cpoly}[2][\polyvar]{\poly{\cpolynotat_{#2}}{#1}}
\newcommand{\divpoly}[1][\polyvar]{\poly{\divpolynotat}{#1}}
\newcommand{\hpoly}[2][\polyvar]{\poly{\hpolynotat_{#2}}{#1}}
\newcommand{\hpolyplain}[1][\polyvar]{\poly{\hpolynotat}{#1}}

%\newcommand{\totalepsilon}{2.1\cdot10^{-5}\xspace}
%\newcommand{\totalepsilon}{2.3\cdot10^{-8}\xspace}
\newcommand{\totalerrori}{4.5\cdot10^{-6}\xspace}
\newcommand{\totalerrorii}{2.8\cdot10^{-9}\xspace}
\newcommand{\blerror}{1.9\cdot10^{-6}\xspace}
\newcommand{\commiterror}{7.4\cdot10^{-12}\xspace}
\newcommand{\deltaerror}{6.4\cdot10^{-7}\xspace}
\newcommand{\pcpkappag}{2.6\cdot10^{-6}\xspace}

% pepper errors
\newcommand{\pcpkappa}{7/9}
\newcommand{\pepperpcperror}{2.3\cdot10^{-8}\xspace}
% includes commitment error: 1000*2*\epsilon_3
\newcommand{\peppertotalerror}{2.4\cdot10^{-8}\xspace}

\newcommand{\hcost}{h}
\newcommand{\dcost}{d}
\newcommand{\ecost}{e}
\newcommand{\fcost}{f}
\newcommand{\foptcost}{f_{lazy}}
\newcommand{\fdivcost}{f_{\textit{div}}}
%\newcommand{\fmodcost}{f_{\textit{mod}}}
% MW: above is b.s. see comment in eval. the concept is simple: field
% multiplication, end of story. implying that there's a notion of field
% multiplication APART from f_mod just makes no sense for this model.
% moreover, our other papers have always assumed a baseline of GMP, not
% native CPU, and we always make that clear.
\newcommand{\fmodcost}{\fcost}
\newcommand{\rcost}{c}
\newcommand{\randcost}{c}
\newcommand{\provervec}{w}
\newcommand{\linfunc}{\phi}
\newcommand{\degree}{D}

\newcommand{\polylog}[1]{\mathrm{polylog}(#1)}
\newcommand{\Interp}[1]{I(#1)}

\newcommand{\Kiirel}{=}

\newcommand{\lengthenc}{\xi}

\newcommand{\eps}{\epsilon}

\newcommand{\totalepsilon}{\eps_G}
\newcommand{\commitepsilon}{\eps_C}
\newcommand{\ssepsilon}{\eps_S}
\newcommand{\bindingepsilon}{\eps_B}
\newcommand{\linearityepsilon}{\eps_L}
\newcommand{\blepsilon}{\eps_F}

\newcommand{\errori}{\epsilon_1}
\newcommand{\errorii}{\epsilon_2}
\newcommand{\erroriii}{\epsilon_3}

\newcommand{\fiso}{f_{iso}}

\newcommand{\prop}{\Phi}
\newcommand{\proppsi}{\Phi_{\Psi}}
\newcommand{\spec}{$\mathscr{S}$\xspace}
\newcommand{\mspec}{\mathscr{S}\xspace}

\newcommand{\equalszero}{\textsc{equals-zero}\xspace}
\newcommand{\notequalszero}{\textsc{not-equals-zero}\xspace}

\newcommand{\ttlt}{\texttt{<}}
\newcommand{\ttleq}{\texttt{<=}}
\newcommand{\ttgt}{\texttt{>}}
\newcommand{\ttgeq}{\texttt{>=}}
\newcommand{\ttne}{\texttt{!=}}
\newcommand{\tteq}{\texttt{==}}

\newcommand{\compilei}{C1\xspace}
\newcommand{\compileii}{C2\xspace}
\newcommand{\compileiii}{C3\xspace}

\newcommand*{\circfont}{\fontfamily{phv}\selectfont}
\newcommand{\tobject}{\texttt{TObject}\xspace}
\newcommand{\tobjects}{\texttt{TObject}s\xspace}
\newcommand{\ranks}{\texttt{ranks}\xspace}
\newcommand{\locks}{\texttt{locks}\xspace}
\newcommand{\lockable}{lockable item\xspace}
\newcommand{\lockables}{lockable items\xspace}
\newcommand{\Lockable}{Lockable item\xspace}
\newcommand{\Lockables}{Lockable items\xspace}
\newcommand{\xmark}{$false$\xspace}
\newcommand{\cmark}{$true$\xspace}




\tikzset{%
  longcircle/.style={%
    shape=rounded rectangle,
    rounded rectangle arc length=180,
    inner sep=+.3em,
    text depth=+.1ex}}

\newcommand{\longcircled}[5][]{%
  \resizebox{1.4em}{0.65em}{%
  \tikz[baseline=#5]{%
  \node[longcircle, draw=#4, fill=#3, text=white, #1]{\,\,\circfont #2\,\,};
  }}}

\newcommand{\leftcirc}[5][]{%
  \resizebox{0.9em}{!}{%
  \tikz[baseline=#5]{%
  \begin{scope}
  \clip(-1.1em,-0.7em) rectangle (0.4em,0.7em);
  \node[longcircle, draw=#4, fill=#3, text=white, #1]{\circfont #2};
  \end{scope}}}\,
	}

\newcommand{\rightcirc}[5][]{%
  \resizebox{0.9em}{!}{%
  \tikz[baseline=#5]{%
  \begin{scope}
  \clip(-0.4em,-0.7em) rectangle (1.1em,0.7em);
  \node[longcircle, draw=#4, fill=#3, text=white, #1]{\circfont #2};
  \end{scope}}}\,
}


\newcommand{\ccircled}[3][]{%
  \longcircled{#2}{#3}{#3}{-4pt}\hspace{-0.2em}
}

\newcommand{\rcirc}[3][]{%
  \rightcirc{\,\,\,#2}{#3}{#3}{-4pt}\kern -0.4em
}

\newcommand{\trcirc}[3][]{%
  \rightcirc{\,\,\,#2}{#3}{#3}{-6pt}
}

\newcommand{\brcirc}[3][]{%
  \rightcirc{\,\,\,#2}{#3}{#3}{-2pt}
}

\newcommand{\lcirc}[3][]{%
  \leftcirc{#2\,\,\,}{#3}{#3}{-4pt}\kern - 0.3em
}

\newcommand{\tlcirc}[3][]{%
  \leftcirc{#2\,\,\,}{#3}{#3}{-6pt}
}

\newcommand{\blcirc}[3][]{%
  \leftcirc{#2\,\,\,}{#3}{#3}{-2pt}
}


% MW: our "house style" is not to italicize "et al". (we also don't
% italicize "i.e." and "e.g.".) rationale: why draw attention to it?
% italics mean "from a foreign language", but is it necessary to
% emphasize the Latin-ness of "et al."?
\newcommand{\etal}{et al.\xspace}
\newcommand{\smcpit}[1]{#1}

\makeatletter
\def\imod#1{\allowbreak\mkern10mu({\operator@font mod}\,\,#1)}
\makeatother

\makeatletter
\setlength{\@fptop}{0pt}
\makeatother


%\renewcommand{\vec}[1]{\mathbf{#1}}

%\def\compactify{\leftmargin=\parindent \itemsep=0.01pt \topsep=0.01pt \partopsep=0pt \parsep=0.01pt}
\def\compactify{\itemsep=0in \topsep=2pt \parsep=0.00in \partopsep=0pt
\leftmargin=2em}
\let\latexusecounter=\usecounter
\newenvironment{CompactItemize}
  {\def\usecounter{\compactify\latexusecounter}
   \begin{itemize}\addtolength{\itemsep}{-0.075in}}
  {\end{itemize}\let\usecounter=\latexusecounter}
\newenvironment{CompactEnumerate}
  {\def\usecounter{\compactify\latexusecounter}
   \begin{enumerate}}
  {\end{enumerate}\let\usecounter=\latexusecounter}

\newenvironment{myitemize}%
  {\begin{list}{\labelitemi}{\itemsep3pt \topsep3pt \parsep0.00in
  \partopsep=3pt \leftmargin1.2em}}%
  {\end{list}}
\newenvironment{myitemize2}%
  {\begin{list}{\labelitemi}{\itemsep1pt \topsep2pt \parsep0.00in
  \partopsep=0pt \leftmargin1.2em}}%
  {\end{list}}
\newenvironment{myitemize4}%
  {\begin{list}{\labelitemi}{\itemsep2pt \topsep2pt \parsep0.00in
  \partopsep=0pt \leftmargin1.2em}}%
  {\end{list}}
\newenvironment{myitemize5}%
  {\begin{list}{\threequartdash}{\itemsep3pt \topsep3pt \parsep0.00in
  \partopsep=3pt \leftmargin1.5em}}%
  {\end{list}}

%\newenvironment{myitemize}%
%  {\begin{list}{\labelitemi}{\itemsep4pt \topsep10pt \parsep0.00in
%  \partopsep=0pt}}%
%  {\end{list}}
\newenvironment{myenumerate}
  {\def\usecounter{\compactify\latexusecounter}
   \begin{enumerate}}
  {\end{enumerate}\let\usecounter=\latexusecounter}

\def\compactsortof{\itemsep=0in \topsep=2pt \parsep=0.00in \partopsep=0pt
\leftmargin=1.7em}
\newenvironment{myenumerate2}
  {\def\usecounter{\compactsortof\latexusecounter}
   \begin{enumerate}}
  {\end{enumerate}\let\usecounter=\latexusecounter}

%\def\compactsortof{\itemsep=3pt \topsep3pt \parsep=0ex \partopsep=0pt
%\leftmargin=1.55em}
\newenvironment{myenumerate3}
  {\def\usecounter{\compactsortof\latexusecounter}
   \begin{enumerate}}
  {\end{enumerate}\let\usecounter=\latexusecounter}

\def\compactsqueeze{\itemsep=0pt \topsep0pt \parsep=0ex \partopsep=0pt
\leftmargin=1.63em}
\newenvironment{myenumerate4}
  {\def\usecounter{\compactsqueeze\latexusecounter}
   \begin{enumerate}}
  {\end{enumerate}\let\usecounter=\latexusecounter}


\newcounter{saveenumi}

\newcommand{\astskip}{\smallskip\noindent\parbox{\linewidth}
			{\center*\hspace{2.5em}*\hspace{2.5em}*\medskip\smallskip}}

% uncomment to use regular paragraphs
%\def\normalpar{}

\ifx\normalpar\undefined
  \newcommand{\mypar}[1]{\textbf{#1}}
\else
  \newcommand{\mypar}[1]{\paragraph{#1}}
\fi

\def\discretionaryslash{\discretionary{/}{}{/}}
{\catcode`\/\active
\gdef\URLprepare{\catcode`\/\active\let/\discretionaryslash
        \def~{\char`\~}}}%
\def\URL{\bgroup\URLprepare\realURL}%
\def\realURL#1{\tt #1\egroup}%

\newcommand{\dbupdate}{\textsc{update}}
\newcommand{\dbselect}{\textsc{select}}
\newcommand{\dbwhere}{\textsc{where}\xspace}
\newcommand{\dbinsert}{\textsc{insert}}
\newcommand{\dbdelete}{\textsc{delete}}
\newcommand{\dbcreate}{\textsc{create}}
\newcommand{\dbdrop}{\textsc{drop}}
\newcommand{\dbindex}{\textsc{index}}
\newcommand{\dbcf}{\textsc{columnfamily}}
\newcommand{\dbtruncate}{\textsc{truncate}}
\newcommand{\dbbatch}{\textsc{batch}}

\newcommand\IN{$\in$}
\newcommand\SIGMA{$\Sigma$}
\newcommand\CUPEQ{$\stackrel{\cup}{=}$}
\newcommand\OLDER {$\rightsquigarrow$}
\newcommand\DEFEQ{$\stackrel{\Delta}{=}$}
\newcommand\SUBI{\textsubscript{i}}
\newcommand\SUBA{\textsubscript{a}}
\newcommand\SUBB{\textsubscript{b}}
\newcommand\SUBONE{\textsubscript{1}}
\newcommand\SUBTWO{\textsubscript{2}}
\newcommand\SUPT{\textsuperscript{T}}
\newcommand\SUPP{\textsuperscript{p}}
\newcommand\SUPI{\textsuperscript{i}}
\newcommand\SUPD{\textsuperscript{d}}


\newcommand\LAR{$\leftarrow$}
\newcommand\RAR{$\rightarrow$}
\newcommand\UAR{$\Uparrow$}
\newcommand\RARONE{${\stackrel{1}{\rightarrow}}$}
\newcommand\RARTWO{${\stackrel{2}{\rightarrow}}$}
\newcommand\RARTHREE{${\stackrel{3}{\rightarrow}}$}
\newcommand\RARK{${\stackrel{k}{\rightarrow}}$}
\newcommand\RARI{${\stackrel{i}{\rightarrow}}$}
\newcommand\RARD{${\stackrel{d}{\rightarrow}}$}
\newcommand\IIE{${\stackrel{\mathfrak{ii}}{\mbox{---}}}$}


\newcommand\SRZT{the serialization graph}
\newcommand\SRZP{$\mathcal{S}^p$}
\newcommand\SRZ{serialization graph}
\newcommand\DEPT{$dep$}
\newcommand\DEPP{$\mathcal{D}^p$}
\newcommand\DEP{$\mathcal{D}$}
\newcommand\CHP{$\mathcal{C}$}
\newcommand\TRAR{${\leadsto}$}
%\newcommand\AFNRAR{${\looparrowright}$}
%\newcommand\AFNRAR{${\mathrlap{++}\rightarrow}$}

\newcommand\ORAR{${\mathrlap{o}\rightarrow}$}
\newcommand\NRAR{$\nrightarrow$}
\newcommand\TRARI{${\stackrel{i}{\leadsto}}$}
\newcommand\TRARD{${\stackrel{d}{\leadsto}}$}
\newcommand\TSRAR{${\stackrel{\epsilon}\leadsto}$}
\newcommand\SRAR{${\mathrlap{*}\rightarrow}$}


\hyphenation{ra-tionale pseudo-constraint}

%%
%% Figure placeholder macros
%%

\definecolor{placeholderbg}{rgb}{0.85,0.85,0.85}
\newcommand{\placeholder}[1]{%
\fcolorbox{black}{placeholderbg}{\parbox[top][1.5in][t]{0.95\columnwidth}{#1}}}


